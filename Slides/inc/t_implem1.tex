\section{Le langage F-Safe}

\begin{frame}
  \frametitle{F-Safe}
  \begin{block}{F-Safe est...}
    \begin{itemize}
      \item Un langage de modélisation
      \item Fonctionnel pur
      \item Terminant
      \item Typé statiquement
      \item A types inductifs
      \item A types polymorphes
    \end{itemize}
  \end{block}
\end{frame}

\begin{frame}
  \frametitle{Syntaxe concrète}
  \begin{block}{Expressions}
    $a, b ::= x$ \\
    \hspace{1,2cm}| $C(a_1, ..., a_n)$ \\
    \hspace{1,2cm}| $C[\tau_1, ..., \tau_m](a_1, ..., a_n)$ \\
    \hspace{1,2cm}| \textbf{fun} $(a_1:\tau_1, ..., a_n:\tau_n) : \tau \Rightarrow a$ \\
    \hspace{1,2cm}| $a(a_1, ..., a_n)$ \\
    \hspace{1,2cm}| \textbf{let} $(x_1:\tau_1 = a_1, ..., x_n:\tau_n = a_n) \{ a \}$ \\
    \hspace{1,2cm}| \textbf{case} $a_1, ..., a_m \{ | f_1 \Rightarrow b_1 | ... | f_n \Rightarrow b_n \}$
  \end{block}
\end{frame}

%\section{Composants}

%\begin{frame}
%  \frametitle{Diagramme de composants}
%  TODO
%\end{frame}

\section{Algorithme de terminaison}

\subsection{Principes}

\begin{frame}
  \frametitle{}
  \begin{block}{Idées}
    \begin{itemize}
      \item Identifier un ordre bien fondé sur les arguments de fonctions
      \item Construire un graphe d'appels pour les fonctions
        mutuellement récursives
      \item Détecter des cycles dans le graphe d'appels
    \end{itemize}
  \end{block}
\end{frame}

\subsection{Ordre bien fondé}

\begin{frame}
  \frametitle{Ordre bien fondé}
  \begin{block}{Relations}
    3 types de relation :
    \begin{itemize}
    \item \textbf{<} : ``inférieur à''
    \item \textbf{=} : ``égal à''
    \item \textbf{?} : incomparables (peut être une relation ``inférieur à'', ``égal'', ou ``supérieur à'')
    \end{itemize}
  \end{block}
  \begin{block}{Loi de déconstruction}
    Soient $x$ et $y$ des variables, et $C$ un constructeur. Si $x = C(y)$, alors on déduit $y \textbf{<} x$.
  \end{block}
\end{frame}

\begin{frame}
  \frametitle{Algorithme}
  \begin{block}{Structure de données}
    \begin{itemize}
    \item Arbre de noms de variables
    \end{itemize}
  \end{block}
  \begin{block}{Algorithme}
    Parcours du corps de la fonction :
    \begin{itemize}
      \item Lors d'une déconstruction ($x = C(y)$), on ajoute le lien de parenté entre les variables ($y < x$)
      \item Lors d'un appel de fonction, on vérifie les parentés entre les variables dans l'arbre
    \end{itemize}
  \end{block}
\end{frame}

\subsection{Matrices de paramètres}

\begin{frame}
  \frametitle{}
  \begin{block}{}
    \begin{itemize}
      \item Construction de la matrice des arguments
    \end{itemize}
  \end{block}
  \begin{block}{Structure de données}
  \end{block}
  \begin{block}{Limites}
    \begin{itemize}
      \item Nécessité de conserver l'ordre des arguments d'une fonction récursive
    \end{itemize}
  \end{block}
\end{frame}

\subsection{Callgraph}
\begin{frame}
  \frametitle{Construction du graphe d'appels}
  \begin{block}{Objectif}
    Représenter tous les appels de fonctions pour détecter d'éventuels appels mutuellement récursifs
  \end{block}
  \begin{block}{Structures de données}
    \begin{itemize}
    \item Le graphe : une Map qui associe un nom de fonction (appelante) à une liste d'arcs sortants
    \item Un arc : la fonction appelée et les noms des paramètres
      ainsi que la matrice associé
    \item Une table pour retenir les fonctions traitées et ne pas parcourir l'AST à l'infini
    \end{itemize}
  \end{block}
\end{frame}

\begin{frame}
  \frametitle{Détection de cycles et terminaison}
  \begin{block}{Un Cycle}
    Un cycle est une liste de couple
    \begin{itemize}
    \item nom de la fonction appelante 
    \item arc du callgraph de la fonction appelante vers fonction appelée
    \end{itemize}
  \end{block}
  \begin{block}{Détection de cycles}
    \begin{itemize}
    \item On parcours le graphe d'appel d'une fonction, en gardant un
      historique des fonction appelés
    \item Si on retombe sur une fonction de l'historique, on ajoute le
      cycle
    \item Une fois tout les cycles directs obtenus, refaire une passe
      pour composer les cycles entre eux
    \end{itemize}
  \end{block}
\end{frame}
  
\section{Vers un langage ``fonctionnel''}

\subsection{Typage}

\begin{frame}
  \frametitle{}
  \begin{block}{}
  \end{block}
\end{frame}

\subsection{Interprète}

\begin{frame}
  \frametitle{}
  \begin{block}{}
  \end{block}
\end{frame}
