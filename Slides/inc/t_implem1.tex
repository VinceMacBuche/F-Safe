\section{Le langage F-Safe}

\begin{frame}
  \frametitle{F-Safe}
  \begin{block}{F-Safe est...}
    \begin{itemize}
      \item Un langage de modélisation
      \item Fonctionnel pur
      \item Terminant
      \item Typé statiquement
      \item A types inductifs
      \item A types polymorphes
    \end{itemize}
  \end{block}
\end{frame}

\begin{frame}
  \frametitle{Syntaxe concrète}
  \begin{block}{Expressions}
    $a, b ::= x$ \\
    \hspace{1,2cm}$C(a_1, ..., a_n)$ \\
    \hspace{1,2cm}$C[\tau_1, ..., \tau_m](a_1, ..., a_n)$ \\
    \hspace{1,2cm}\textbf{fun} $(a_1:\tau_1, ..., a_n:\tau_n) : \tau \Rightarrow a$ \\
    \hspace{1,2cm}$a(a_1, ..., a_n)$ \\
    \hspace{1,2cm}\textbf{let} $(x_1:\tau_1 = a_1, ..., x_n:\tau_n = a_n) \{ a \}$ \\
    \hspace{1,2cm}\textbf{case} $a_1, ..., a_m \{ | f_1 \Rightarrow b_1 | ... | f_n \Rightarrow b_n \}$
  \end{block}
\end{frame}

\section{Composants}

\begin{frame}
  \frametitle{Diagramme de composants}
  TODO
\end{frame}

\section{Algorithme de terminaison}

\subsection{Principes}

\begin{frame}
  \frametitle{}
  \begin{block}{Idées}
    \begin{itemize}
      \item Identifier un ordre bien fondé sur les arguments de fonctions
      \item Construire un graphe d'appels pour les fonctions
        mutuellement récursives
    \end{itemize}
  \end{block}
\end{frame}

\subsection{Ordre bien fondé}

\begin{frame}
  \frametitle{Ordre bien fondé}
  \begin{block}{Relations}
    3 types de relation :
    \begin{itemize}
    \item \textbf{<} : ``inférieur à''
    \item \textbf{=} : ``égal à''
    \item \textbf{?} : incomparables (peut être une relation ``inférieur à'', ``égal'', ou ``supérieur à'')
    \end{itemize}
  \end{block}
  \begin{block}{Loi de déconstruction}
    Soient $x$ et $y$ des variables, et $C$ un constructeur. Si $x = C(y)$, alors on déduit $y \textbf{<} x$.
  \end{block}
\end{frame}

\begin{frame}
  \frametitle{Algorithme}
\end{frame}

\subsection{Matrices de paramètres}

\begin{frame}
  \frametitle{}
  \begin{block}{}
    \begin{itemize}
      \item Construction de la matrice des arguments
      \item Exemple ?
    \end{itemize}
  \end{block}
  \begin{block}{Structure de données}
    TODO
  \end{block}
\end{frame}

\subsection{Callgraph}

\begin{frame}
  \frametitle{}
  \begin{block}{}
  \end{block}
\end{frame}

\section{Vers un langage ``fonctionnel''}

\subsection{Typage}

\begin{frame}
  \frametitle{}
  \begin{block}{}
  \end{block}
\end{frame}

\subsection{Interprète}

\begin{frame}
  \frametitle{}
  \begin{block}{}
  \end{block}
\end{frame}
